\section*{Заключение}
\addcontentsline{toc}{section}{Заключение}
\label{sec:Conclusion} \index{Conclusion}
\large

\begin{itemize}
\item Выполнена модернизация алгоритма.
\item Сформулирована гипотеза вероятностной сложности алгоритма.

Разработан алгоритм с вероятностной сложностью:

$$\approx O(n^2(\frac{e}{2})^{\ln(n)^2} \ln(n))$$

Алгоритм является универсальным и может применяться для решения большого количества задач (например, задач указанных в пункте 4.1 дипломной работы).

\item Реализовано 2 програмы: с графическим интерфейсом для удобного использования, с консольным интерфейсом для запуска на суперкомпьютере.

Написанная программа с интерфейсом имеет масштабируемый функционал. На данный момент поддерживает функционал решения следующих задач (при $n \leq  500$):

\begin{itemize}
\item Поиск автоморфизмов
\item Поиск изоморфизмов
\item Поиск гомоморфизмов
\item Решение задачи Коши в подстановках
\end{itemize}

\item Проведены опыты на суперкомпьютере <<Ломоносов>>.

Получены оценки сложности алгоритма при больших размерах графа (количество вершин $n$ > 1000).

\end{itemize}
