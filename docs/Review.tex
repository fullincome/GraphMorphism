\documentclass[oneside,final,12pt,a4paper]{extreport}

\usepackage[utf8]{inputenc}
\usepackage{rotating} 

\usepackage[english, russian]{babel}
\usepackage{listings}
\usepackage{amsfonts} % Пакеты для математических символов и теорем
\usepackage{amstext}
\usepackage{amssymb}
\usepackage{amsthm}
\usepackage{graphicx} % Пакеты для вставки графики
\usepackage{subfig}
\usepackage{color}
\usepackage[unicode]{hyperref} 
\usepackage[nottoc]{tocbibind} % Для того, чтобы список литературы отображался в оглавлении
\usepackage{algorithmic} % Для записи алгоритмов в псевдокоде
\usepackage{algorithm}
\usepackage{verbatim} % Для вставок заранее подготовленного текста в режиме as-is
\usepackage{indentfirst}
\pagestyle{empty}





\begin{document}
\begin{center}
{
\large
МОСКОВСКИЙ ГОСУДАРСТВЕННЫЙ УНИВЕРСИТЕТ имени М.В.Ломоносова
\newline

ФАКУЛЬТЕТ ВЫЧИСЛИТЕЛЬНОЙ МАТЕМАТИКИ
И КИБЕРНЕТИКИ
}
\newline

\textbf{Отзыв научного руководителя на выпускную квалификационную работу}

\textbf{Ефремова Степана Сергеевича}

\textbf{<<Разработка итерационных алгоритмов поиска автоморфизмов и изоморфизмов комбинаторных объектов>>}

\end{center}

В работе рассматривается решение задачи поиска автоморфизмов графов с помощью итерационного алгоритма. Как выяснилось, сложность разработанного алгоритма превосходит по скорости все известные на данный момент итерационные алгоритмы, которые можно применять для любых графов.

Основные результаты работы следующие. Выполнена модернизация алгоритма. Разработан алгоритм с вероятностной сложностью:

$O(n^2(\frac{e}{2})^{\ln(n)^2} \ln(n))$.
Алгоритм является универсальным и может применяться для решения большого количества задач.

Реализовано 2 програмы: с графическим интерфейсом для удобного использования, с консольным интерфейсом для запуска на суперкомпьютере
Написанная программа с интерфейсом имеет масштабируемый функционал и на  данный момент поддерживает функционал решения следующих задач: поиск автоморфизмов, изоморфизмов, гомоморфизмов, решение задачи Коши в подстановках.

Проведены опыты на суперкомпьютере <<Ломоносов>>. Получены оценки сложности алгоритма при больших размерах графа (количество вершин $n$ > 1000).


Работа представляет собой полноценное исследование алгоритма, с практическим применением реализованных программ. Результаты работы, изложенные выше, получены автором самостоятельно, на основе существующего алгоритма. Рекомендуемая оценка <<отлично>>.

\begin{flushleft}
\textit{Доцент, кандидат физико-математических наук}
\newline

\textit{В.Н.Егоров}
\end{flushleft}

\end{document}
