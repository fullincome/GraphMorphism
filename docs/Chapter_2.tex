\section{Основные определения и обозначения}
\label{sec:Definition_2} \index{Definition_2}
\large

\subsection{Определения}

В работе часто упоминается понятие изоморфизма графов, которое описано ниже. В общем случае, \textbf{изоморфизм} можно описать так: обратимое отображение (биекция) между двумя множествами, наделёнными структурой, называется изоморфизмом, если оно сохраняет эту структуру.

\begin{itemize}
\item \textbf{Автоморфизм алгебраической системы} -- изоморфизм, отображающий алгебраическую систему на себя.

\item \textbf{Автоморфизм графа} -- отображение множества вершин на себя, сохраняющее смежность.

\item \textbf{Изоморфизм графов $G=\left\langle V_{G},E_{G}\right\rangle$ и $H=\left\langle V_{H},E_{H}\right\rangle$} -- биекция между множествами вершин графов $f\colon \ V_{G}\rightarrow V_{H}$ такая, что любые две вершины $u$ и $v$ графа $G$ смежны тогда и только тогда, когда вершины $f(u)$ и $f(v)$ смежны в графе $H$.

\item \textbf{Инвариант графа} -- некоторое числовое значение или упорядоченный набор значений (хэш-функция), характеризующее структуру графа $ G=\langle A,V\rangle $  и не зависящее от способа обозначения вершин или графического изображения графа.

\item \textbf{Регулярный (однороодный) граф} -- граф, степени всех вершин которого равны, то есть каждая вершина имеет одинаковое количество соседей.

\item \textbf{Cильно регулярный граф $srg(v, k, \lambda, \mu)$} -- регулярный граф $G=\left\langle V_{G},E_{G}\right\rangle$ с $v$ вершинами и степенью $k$, для которого существуют целые $\lambda$ и $\mu$ такие, что:
\begin{enumerate}
\item Любые две смежные вершины имеют $\lambda$ общих соседей.
\item Любые две несмежные вершины имеют $\mu$ общих соседей.
\end{enumerate}
Приложение такое то.


\end{itemize}




\subsection{Обозначения}

\begin{itemize}
\item $A = A^{n\times n}$ - матрица смежности графа $G(V,E)$ размера $n\times n$.

\item $a_{ij}$ - элементы матрицы $A$, $\ i,j = 1,\ldots,n=|V|$.

\item Подстановка изоморфизма - биекция $f$ вида:

\[ 
    \begin{pmatrix}
    {
		a_1 & a_2 & \ldots & a_n \cr
		f(a_1) & f(a_2) & \ldots & f(a_n) 
	}
    \end{pmatrix}
\]

\item Частичная подстановка изоморфизма (частичное отображение) - подстановка вида: 

\[
    \begin{pmatrix}
    {
		a_1 & a_2 & \ldots & a_k \cr
		f(a_1) & f(a_2) & \ldots & f(a_k)
    }
    \end{pmatrix}
\]

, где $k \leq n$.
%, где $r_1,\ldots,r_k$ - фиксированные (заданные) элементы, $q_{k+1},\ldots,q_n$ - произвольные, причем последовательность $r_1,\ldots,r_k$, $q_{k+1},\ldots,q_n$ является перестановкой вершин заданного графа.

\item $M_k$ - множество, состоящее из элементов $g^s_k$, $\ s = 1,\ldots,n_k=|M_k|$. 

\item $|M_k|$ - количество элементов $g^s_k$ в множестве.

\item $M'_k$ - множество элементов, которые содержатся в $M_k$ и удовлетворяют критерию $h$.

\item h - критерий (описанный в разделе 2.2).

\item $\{M'_k\}$ - последовательность множеств $M'_k$, $k = 1,\ldots,n$. 
\end{itemize}


