\section{Структура алгоритма}
\label{sec:AlgoStruct_3} \index{AlgoStruct_3}
\large

\subsection{Входные данные}

На вход алгоритму подается матрица смежности $A^{n \times n}$ графа $G(V,E)$, $n = |V|$.

\subsection{Процедура построения множеств частичных отображений}

Для матрицы $A$ размера $n \times n$ строится последовательность множеств частичных отображений ${M'_n}$ следующим образом. $M_0 = M'_0 = \{g_0\}$, $g_0$ - пустая подстановка. Для всех $i = 1 \ldots n$:
\begin{enumerate}
\item Фиксируем подматрицу $A_i$ размера $i \times i$.

\item Строим множество $M_i$:
\begin{enumerate}
\item Рассматриваем все возможные частичные подстановки $g_i$, полученные из подстановок $g^s_{i-1}$ путем добавления в них отображения $f(u_i) = v_l$.
\item Генерируем множество $M_i$ из подстановок $g_i$, для которых в отображении $f(u_i) = v_l$:
$$l \in \{l\ |\ v_l \neq f(u_j), j = 1 \ldots i-1\}$$
\end{enumerate}

\item Строим множество $M'_i$ (для этого убираем из множества $M_i$ элементы $g_i$, которые не удовлетворяют критерию):
\begin{enumerate}
\item Применяем частичные подстановки к матрице $A$ по отдельности: $\widehat{g_i}A\widehat{g_i}^{-1} = A^{i \times i}_{g_i}$.
\item Генерируем множество $M'_i$ из частичных подстановок, для которых выполнено: $A^{i \times i}_{i} = A^{i \times i}_{g_i}$.
\end{enumerate}

\item Если $i = n$, процедура завершена.
\end{enumerate}



Таким образом, получается последовательнось множеств частичных отображений:

$$\{M'_k\}: M'_1 \supseteq M'_2 \supseteq \ldots \supseteq M'_n$$.

Множество $M'_n$ является множеством всех возможных частичных подстановок $g_n$ длины $n$, то есть всех подстановок изоморфизма графа $G$ на себя. Следовательно $M'_n$ содержит все автоморфизмы.


\subsection{Выходные данные}

Множество подстановок изоморфизма $g^s_n$ графа $G$ на себя (автоморфизмы).
Для всех $s = 1, \ldots ,|M'_n|$, $g^s_n$ имеет вид:

\[ 
    \begin{pmatrix}
    {
		u_1 & u_2 & \ldots & u_n \cr
		f_s(u_1) & f_s(u_2) & \ldots & f_s(u_n) 
	}
    \end{pmatrix}
\]

\subsection{Примечания к процедуре}

В случае задачи определения изоморфизма (или гомоморфизма) графов на вход падается матрица смежности $A^{n \times n}$ и матрица смежности $B^{m \times m}$. Не ограничивая общности считаем, что $n \geq m$. Тогда отличия в процедуре следующие:

\begin{itemize}
\item Случай изоморфизма: вместо подматрицы $A_i$ используется подматрица $B_i$.

\item Случай гомоморфизма: вместо подматрицы $A_i$ используется подматрица $B_i$. Процедура завершается при $i = m$.
\end{itemize}


Критерием $h$ является проверка подматриц на равенство. Предположим, требуется определить, удовлетворяет ли элемент $g^s_k$ критерию. Для этого необходимо, чтобы элементы $a_{ij}$, для всех $i,j \leq k$, были равны соответствующим элементам $a_{r_ir_j}$. Подматрица $\widehat{g^s_k}A\widehat{g^s_k}^{-1} = A_{g^s_k}$ выглядит так:

\[ \bordermatrix{
& r_1 & r_2 & \dots & r_k \cr
r_1 & a_{r_1r_1} & a_{r_1r_2} & \dots & a_{r_1r_k} \cr
r_2 & a_{r_2r_1} & a'_{r_2r_2} & \dots & a_{r_2r_k} \cr
\vdots & \vdots & \vdots & \ddots & \vdots \cr
r_k & a_{r_kr_1} & a_{r_kr_2} & \dots & a_{r_kr_k} \cr}
\]

Если хотя бы одно из равенств не выполнено, это означает, что после применения частичной подстановки к $A$ подматрица $A_i$ не совпадает с $A_{g^s_k}$. Но добавление отображения в частичную подстановку не изменяют подматрицу $A_{g^s_k}$, следовательно для $g^s_n$ образованной от $g^s_k$ $A_n \neq A_{g^s_n}$, ни для какого $s$. Это значит, что частичную подстановку $g^s_k$ можно не рассматривать.


