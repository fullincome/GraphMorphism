\section{Модернизация}
\label{sec:AlgoModern_5} \index{AlgoModern_5}
\large

\subsection{Оптимизация по времени работы}
На основе результатов тестирования программы и анализа выяснилось, что эффективность алгоритма тесно связана с тем, как нумеруются вершины графа. Другими словами, время работы программы зависит от того, какой матрицей смежности (из многих) представляется граф.

Модернизация заключается в том, что на каждом этапе можно требовать, чтобы мощность множества частичных отображений была минимальна. Однако представить матрицу смежности нужным образом не представляется возможным при больших размерах, и время работы, затрачеваемой на это, превышает время работы алгоритма. Например, при $n = 100$ потребуется всего лишь 100 операций, чтобы выяснить, с какой вершины эффективнее всего начать отсчет на первой итерации. Но для того, чтобы получить выгоду на второй итерации, уже требуется более 10 000 операций [3]. Поэтому оптимальным является уменьшить только начальное множество частичных отображений.

Для получения первого множества, мощность которого будет наименьшей, в
изначальной матрице меняется порядок строк/столбцов (переименование вершин), а
именно, необходимо поменять строки и соответствующие им столбцы таким образом,
чтобы на месте первого элемента главной диагонали стояло то значение, которое встречается меньше всех других на диагонали. То есть, если на главной диагонали $70\%$ нулей и $30\%$ единиц,
необходимо поместить на первую позицию единицу. Эффективность данной модификации
исходит из того, что первая итерация алгоритма составляет множество из тех номеров
строк (столбцов), в которых значение на главной диагонали совпадает с первым
элементом главной диагонали. Значит, выбрав на эту позицию наименьший по количеству встречаний на диагонали
элемент, получается наименьшее по мощности множество.

Оценки получены для случайных матриц (вероятность нуля и единицы в каждой позиции
одинакова и равна $\frac{1}{2}$).
Предположим, что в матрице $n\times n$ на главной диагонали находится $k$ нулей, где $k \leq \frac{1}{2} n$ (если количество нулей больше половины, то за $k$ обозначается количество единиц).
Тогда, если на первое место главной диагонали выбирать элемент случайным образом,
получим, что математическое ожидание размера полученного множества будет $\frac{k}{n}*k + \frac{n-k}{n}*(n-k)$.
В модифицированном алгоритме всегда будет получаться $k$. Таким образом,
улучшение составляет $\frac{\frac{k}{n} * k + \frac{(n-k)}{n} * (n-k)} {k} = 2 \frac{k}{n} + \frac{n}{k} - 2$ раз. В случае диагонали,
состоящей из одних нулей (единиц), то есть k = 0, улучшение равняется 1 (мнощьность
множеств одинакова).
Для получения полной оценки необходимо посчитать математическое ожидание улучшения для
произвольного числа нулей и единиц на главной диагонали. Вероятноть $k$ нулей
составляет $\frac{C_n^k}{2^n}$. Тогда математическое ожидание улучшения $\frac{1}{2^{n-1}} + 2 \sum\limits_{i=1}^{\lceil\frac{n}{2}\rceil} \frac{C_n^k}{2^n} * (2\frac{k}{n} + \frac{n}{k} - 2)$.
Далее в таблице приведены значения для различных $n$:

\begin{tabular}[t]{||l|l||}
\hline
Количество вершин графа & Коэф. уменьшения мощности \\
\hline
4 &  2.12500 \\
\hline
8 & 2.11198 \\
\hline
14 & 1.69565 \\
\hline
20 & 1.51723 \\
\hline
50 & 1.27697 \\
\hline
100 & 1.18312 \\
\hline
200 & 1.12400 \\
\hline
\end{tabular}

\subsection{Обработка особых случаев}
Из описания алгоритма следует: если в начале построения последовательности множеств частичных отображений, множества имееют размер $\approx n \times (n-1) \times \ldots \times i$ для нескольких $M'_i$, то возможны 2 случая:

\begin{enumerate}
\item Граф действительно имеет много автоморфизмов. В этом случае работа алгоритма будет занимать большее время, чем обычно 

\item Матрица смежности алгоритма оказалась не совсем удобной для использовани. В этом случае можно просто выбрать случайную другую матрицу смежности (например, случай двудольного графа)
\end{enumerate} 

\subsection{Ресурс параллелизма}
Для поиска автоморфизмов матрицы порядка $n$ итерационным алгоритмом в параллельном варианте требуется выполнить:

\begin{itemize}
\item $n$ ярусов сравнений (количество сравнений $ k = 1 \ldots n: \frac{1}{e^{k}} \frac{n^{n+1/2}}{(n-k-2)^{(n-k-3/2)}~2^{(k)^2}} \times (n - k)(2k + 1) $)
\end{itemize}

При классификации по высоте ЯПФ, таким образом, алгоритм имеет сложность $ O(n) $.

При классификации по ширине ЯПФ его сложность $ O(n(\frac{e}{2})^{\ln(n)^2} \ln(n))) $.
