\section*{Аннотация}
\addcontentsline{toc}{section}{Аннотация}
\label{sec:Annotation} \index{Annotation}
\large

Дипломная работа разбита на следующие пункты:
\begin{enumerate}
\item Постановка задачи: сформулированы задачи дипломной работы.
\item Основные определения и обозначения: приводятся основные понятия, используемые в ходе работы.
\item Структура алгоритма: описывается алгоритм, приведенный в статье, который является основой исследования.
\item Исследование алгоритма: в данном пункте описан класс решаемых задач алгоритмом, получены оценки сложности. А также определены способы модернизации, которые удалось найти в ходе исследования. Особое внимание уделяется исследованию расспараллеливания вычислений в программной реализации.
\item Модернизация алгоритма: описаны все найденные способы модернизации.
\item Практическое применение: рассказывается о различных применениях разработанного алгоритма для графов и некоторых других комбинаторных объектов. В том числе, рассматриваются способы использования алгоритма для задачи Коши.
\item Разработка программных модулей: рассматриваются ключевые моменты, на которые уделялось особое внимание при реализации программных модулей. В результате получены 2 программы: упрощенный вариант с графическим интерфейсом для использования на графах с небольшим количеством вершин; вариант реализации для суперкомпьютеров для более удобного сбора информации и исследования эффективности.
\item Результаты опробования программ: в этом пункте приводится анализ результатов, полученных при тестировании разработанных программ на различных данных, также подробно проанализированы результаты запусков программы на суперкомьпьютере Ломоносов.
\end{enumerate}