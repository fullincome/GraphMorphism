\documentclass{beamer}
\usetheme{metropolis}
\usepackage[utf8]{inputenc}
\usepackage[russian]{babel}

\begin{document}

% Конфигурация титульного листа
%---------------------------
\title{Разработка итерационных алгоритмов поиска автоморфизмов
и изоморфизмов комбинаторных объектов.}

\institute{Московский государственный университет имени М.В.Ломоносова\\
Факультет вычислительной математики и кибернетики\\
Кафедра информационной безопасности}

\author{\textit{\underline{Автор:}}\\ Ефремов Степан Сергеевич (419 группа)\\ \\
\textit{\underline {Научный руководитель:}}\\ доцент, к.ф.- м.н.\\ Егоров Владимир Николаевич}

%---------------------------


% 1 слайд - титульная страница
%---------------------------
\begin{frame}{}
\titlepage
\end{frame}
%---------------------------


% 2 слайд - содержание
%---------------------------
\begin{frame}{Содержание}
\tableofcontents
\end{frame} 
%---------------------------


\section{Постановка задачи}
% 3 слайд
%---------------------------
\begin{frame}{Постановка задачи}
\small
\begin{enumerate}
\item Исследование свойств алгоритма:
\begin{itemize}
\scriptsize
\item Определение класса решаемых задач
\item Вероятностная сложность
\item Теоретическая возможность распараллеливания
\end{itemize}
\item Задачи, связанные с реализацией для ПК:
\begin{itemize}
\scriptsize
\item Реализация в виде программы с графическим интерфейсом
\item Эксперименты поиска автоморфизмов на известных графах
\item Поддержа функционала нахождения изоморфного вложения графов
\end{itemize}
\item Задачи, связанные с реализацией для суперкомьютера:
\begin{itemize}
\scriptsize
\item Модернизация алгоритма на основе исследований
\item Реализация в виде программы для запуска на суперкомпьютере
\item Эксперименты поиска автоморфизмов графов на суперкомпьютере
\end{itemize}
\item Исследование практического применения алгоритма для задачи Коши
\end{enumerate}
\end{frame} 
%---------------------------


\section{Существующие решения}
% 5 слайд
%---------------------------
\begin{frame}{Направления исследований}
На данный момент сформированы два направления изучения и решения проблемы поиска изоморфизмов графов:
\begin{itemize}
\item Теоретическое, в котором проблема изоморфизма рассматривается с позиций современной теории сложности алгоритмов и вычислений (подход использует понятие инвариантов графа).

\item Практическое, предполагающее разработку алгоритмов, решающих задачу изоморфизма графов за <<практически приемлемое>> время (направленный перебор).
\end{itemize}
\end{frame}
%---------------------------


% 6 слайд
%---------------------------
\begin{frame}{Сравнение сложностей алгоритмов}

\begin{table}[h]
\centering
\begin{tabular}[t]{|l|l|l|}
\hline
\textbf{Автор} & \textbf{Ограничения} & \textbf{Сложность}\\
\hline
Егоров В.Н., Егоров А.В. & lll & lll\\
\hline
Егоров В.Н., Егоров А.В. & lll & lll\\
\hline
Егоров В.Н., Егоров А.В. & lll & lll\\
\hline
\end{tabular}
\caption{Алгоритмы поиска автоморфизмов}
\end{table}

\end{frame}
%---------------------------


\section{Разработанные решения}
% 7 слайд
%---------------------------
\begin{frame}{numbered lists}
\begin{enumerate}
\item Introduction to  \LaTeX  
\item Course 2 
\item Termpapers and presentations with \LaTeX 
\item Beamer class
\end{enumerate}
\end{frame}
%---------------------------


% 8 слайд
%---------------------------
\begin{frame}{numbered lists with pause}
\begin{enumerate}
\item Introduction to  \LaTeX \pause 
\item Course 2 \pause 
\item Termpapers and presentations with \LaTeX \pause 
\item Beamer class
\end{enumerate}
\end{frame}
%---------------------------


\section{Результаты} 
% 9 слайд
%---------------------------
\begin{frame}{Tables}
\begin{tabular}{|c|c|c|}
\hline
\textbf{Date} & \textbf{Instructor} & \textbf{Title} \\
\hline
WS 04/05 & Sascha Frank & First steps with  \LaTeX  \\
\hline
SS 05 & Sascha Frank & \LaTeX \ Course serial \\
\hline
\end{tabular}
\end{frame}
%---------------------------


% 10 слайд
%---------------------------
\begin{frame}{Tables with pause}
\begin{tabular}{c c c}
A & B & C \\ 
\pause 
1 & 2 & 3 \\  
\pause 
A & B & C \\ 
\end{tabular}
\end{frame}
%---------------------------


\section{Положения, выносимые на защиту}
% 11 слайд
%---------------------------
\begin{frame}{blocs}

\begin{block}{title of the bloc}
bloc text
\end{block}

\begin{exampleblock}{title of the bloc}
bloc text
\end{exampleblock}


\begin{alertblock}{title of the bloc}
bloc text
\end{alertblock}
\end{frame}
%---------------------------


\end{document}