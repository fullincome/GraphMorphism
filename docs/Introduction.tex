\section*{Введение}
\addcontentsline{toc}{section}{Введение}
\label{sec:Introduction} \index{Introduction}
\large 

В дипломной работе рассматривается итерационный алгоритм поиска автоморфизмов и изоморфизмов графов, предложенный в статье В.Н.Егорова \cite{ArtMain_Egorov}.

Проблема изоморфизма графов в настоящее время имеет статус одной из сложных и важных с теоретической точки зрения задач комбинаторной математики. Известно, что решение задачи нахождения изоморфного подграфа с точки зрения алгоритмической сложности позволило бы ответить на вопрос о совпадении или различии классов P и NP. В последние годы сформировались два направления изучения и решения данной проблемы:
\begin{enumerate}
\item Теоретическое, в котором проблема изоморфизма рассматривается с позиций современной теории сложности алгоритмов и вычислений.
\item Практическое, предполагающее разработку алгоритмов, решающих задачу изоморфизма графов за <<практически приемлемое>> время.
\end{enumerate}

Первые серьёзные результаты в теоретическом направлении появились в 50-х и 60-х годах прошлого века. Для нескольких семейств графов была установлена полиномиальная сложность нахождения изоморфизма. В 70-х годах, особенно после выхода в свет замечательной монографии Нормана Биггса <<Алгебраическая теория графов>>, к решению этой задачи подключились крупные специалисты в области линейной алгебры, теории групп и ряда других направлений общей алгебры. Результаты не замедлили сказаться - открыто много новых интересных алгебраических свойств графов. В 1979 году Егорову В.Н. удалось получить положительное решение гипотезы Адама об изоморфизме графов с циркулянтными матрицами смежности вершин порядка свободного от квадратов \cite{ArtCircul_EgorovMarkov}. В 80-х годах усилиями большой группы математиков был получен наилучший на настоящий момент результат о том, что для графов с ограниченной валентностью вершин задача распознавания изоморфизма имеет полиномиальную сложность \cite{ArtBoundValence_Luks}.

Алгоритмы разрабатываются не только профессиональными математиками и программистами, но и людьми, не имеющими фундаментального образования. Известно, что эта тематика в разных модификациях имеет широкий спектр приложений в таких областях:
\begin{enumerate}
\item Криптография.
\item Теория кодирования.
\item Информационные технологии.
\item Химия.
\item Биология.
\end{enumerate}

Следует отметить, что и здесь сложились два принципиально различных подхода. Первый подход использует понятие инвариантов графа, т.е. таких функций от вершин графа или от определённых структур графа, включая и сам граф, которые не меняют своего значения при перестановках вершин графа. Примеров таких инвариантов можно привести достаточно много, причём с их помощью часто удавалось решать самые разнообразные частные задачи. Однако, полной системы инвариантов, которая позволяла бы определить наличие или отсутствие изоморфизма до настоящего времени получить не удалось. Кроме того, слабостью этого подхода является то, что эти инварианты бесполезно применять в случае, если рассматриваемые графы имеют заведомо транзитивные группы автоморфизмов.

Второй подход не связан с вычислением различных характеристик графов. Предлагаемые алгоритмы являются итерационными и определяют наличие или отсутствие изоморфизма методом направленного перебора, основной идеей которого является построение частичных отображений с последующим промежуточным критерием. Понятно, что это одна из версий метода ветвей и границ. Пожалуй, первое упоминание о таком алгоритме содержится в \cite{Book_Reingold} (стр. 397-402). В настоящее время имеется несколько алгоритмов такого рода, которые авторами рекламируются как <<эффективные>>.

Ниже приведены итерационные алгоритмы и их сложности:

\begin{table}[h]
%--------------------------------------
\centering
\begin{tabular}[t]{|l|l|l|}
\hline
\textbf{Авторы} & \textbf{Ограничение} & \textbf{Сложность}\\
\hline
Егоров В.Н., Егоров А.В. \cite{ArtMain_Egorov} & --- & $O(n^2(\frac{e}{2})^{\ln(n)^2} \ln(n))$\\
\hline
Babai \cite{ArtSrg_Babai}& Сильно регулярные графы & O(Exp(2fi log2 n)\\
\hline
Автор 3 & 1.12400 & пум пум\\
\hline
Автор 4 & 1.12400 & пум пум\\
\hline
Автор 5 & 1.12400 & пум пум\\
\hline
Автор 6 & 1.12400 & пум пум\\
\hline
\end{tabular}
%--------------------------------------
\caption{Алгоритмы поиска автоморфизмов}
\label{tabular:algos}
\end{table}
