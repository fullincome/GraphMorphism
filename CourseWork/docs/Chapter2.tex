\section{Постановка задачи}
\label{sec:Chapter2} \index{Chapter2}
\large

Целью курсовой работы является модернизация итерационного алгоритма поиска автоморфизмов графа, предложенного в статье В.Н.Егорова [1] и его реализация в виде программы, поддерживающей графичесий интерфейс для удобного ввода графа и получения итоговых и промежуточных результатов. А также исследование времени работы полученной программы на различных размерах входных данных и определение ограничения на максимально возможное количество вершин вводимого графа.

Автоморфизм графа - есть отображение множества вершин на себя, сохраняющее смежность.
Множество таких автоморфизмов образует вершинную группу графа или просто группу графа [2]. Ставится задача поиска таких групп.

Поставленная задача не является простой, особенно для графов, имеющих большое количество вершин. Для ее решения граф удобно представить в виде матрицы смежности. Пусть, для удобства, элементы этой матрицы состоят из единиц и нулей. Если на пересечении строки $i$ и столбца $j$ стоит $1(0)$, и $i\neq j$ - это означает, что существует (не существует) ребро (i,j), если $i = j$, вершина под номером $i$ имеет (не имеет) петлю. При отображении одной вершины в другую, в матрице смежности меняются местами строки и столбцы соответствующих вершин. Автоморфизмом является следующее отображение: $ \widehat{g} A \widehat{g}^{-1} =A  $, где $ \widehat{g}$ - матричный вид подстановки $g$. Польза такого представления графа заключается в том, что для поиска автоморфизма $g$, являющегося отображением множества вершин графа $G$ на себя, достаточно проверить равенство элементов матрицы смежности до и после применения подстановки $g$. 