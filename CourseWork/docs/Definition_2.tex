\section{Основные определения и обозначения}
\label{sec:Definition_2} \index{Definition_2}
\large

\subsection{Определения}

В работе часто упоминается понятие изоморфизма графов, которое описано ниже. В общем случае, \textbf{изоморфизм} можно описать так: обратимое отображение (биекция) между двумя множествами, наделёнными структурой, называется изоморфизмом, если оно сохраняет эту структуру.

\begin{itemize}
\item \textbf{Автоморфизм алгебраической системы} — изоморфизм, отображающий алгебраическую систему на себя.
\item \textbf{Автоморфизм графа} - отображение множества вершин на себя, сохраняющее смежность.
\item \textbf{Изоморфизм графов $G=\left\langle V_{G},E_{G}\right\rangle$ и $H=\left\langle V_{H},E_{H}\right\rangle$} - биекция между множествами вершин графов $f\colon \ V_{G}\rightarrow V_{H}$ такая, что любые две вершины $u$ и $v$ графа $G$ смежны тогда и только тогда, когда вершины $f(u)$ и $f(v)$ смежны в графе $H$.
\end{itemize}




\subsection{Обозначения}

\begin{itemize}
\item $A = A^{n\times n}$ - матрица смежности графа $G(V,E)$ размера $\times n$; $a_{ij}$ - элементы матрицы ,$\ i,j = 1,\ldots,n=|V|$.

\item Частичным отображением $g^s_k$ ($s$ - обозначает индекс элемента в множестве $M_k$) называется подстановка вида 

\[ \begin{pmatrix} {
		1 & 2 & \ldots & k & k+1 &\ldots & n\cr
		r_1 & r_2 & \ldots & r_k & q_{k+1} & \ldots & q_n}
    \end{pmatrix}
\]

, где $r_1,\ldots,r_k$ - фиксированные (заданные) элементы, $q_{k+1},\ldots,q_n$ - произвольные, причем последовательность $r_1,\ldots,r_k$, $q_{k+1},\ldots,q_n$ является перестановкой вершин заданного графа.

\item $M_k$ - множество, состоящее из элементов $g^s_k$, $\ s = 1,\ldots,n_k=|M_k|$. 

\item $|M_k|$ - количество элементов $g^s_k$ в множестве.

\item $M'_k$ - множество элементов, которые содержатся в $M_k$ и удовлетворяют критерию $h$.

\item h - критерий (описанный в разделе 2.2).

\item $\{M'_k\}$ - последовательность множеств $M'_k$, $k = 1,\ldots,n$. 
\end{itemize}


