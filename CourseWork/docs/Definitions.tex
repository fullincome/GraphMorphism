\section{Основные определения и обозначения}
\label{sec:Definitions} \index{Definitions}
\large

Идея алгоритма заключается в построении последовательности множеств частичных отображений по матрице смежности графа. Последнее множество из этой последовательности будет содержать в себе все автоморфизмы графа. 

Для описания построения такой последовательности будут использоваться следующие определения и обозначения.

\subsection{Определения и обозначения}
$A = A^{n\times n}$ - матрица смежности графа $G(V,E)$ размера $n\times n$; $a_{ij}$ - элементы матрицы ,$\ i,j = 1,\ldots,n=|V|$.

Частичным отображением $g^s_k$ ($s$ - обозначает индекс элемента в множестве $M_k$) называется подстановка вида 

\[ \begin{pmatrix} {
		1 & 2 & \ldots & k & k+1 &\ldots & n\cr
		r_1 & r_2 & \ldots & r_k & q_{k+1} & \ldots & q_n}
    \end{pmatrix}
\]

, где $r_1,\ldots,r_k$ - фиксированные (заданные) элементы, $q_{k+1},\ldots,q_n$ - произвольные, причем последовательность $r_1,\ldots,r_k$, $q_{k+1},\ldots,q_n$ является перестановкой вершин заданного графа.

$M_k$ - множество, состоящее из элементов $g^s_k$, $\ s = 1,\ldots,n_k=|M_k|$. 

$|M_k|$ - количество элементов $g^s_k$ в множестве.

$M'_k$ - множество элементов, которые содержатся в $M_k$ и удовлетворяют критерию $h$.

h - критерий (описанный в разделе 2.2).

$\{M'_k\}$ - последовательность множеств $M'_k$, $k = 1,\ldots,n$. 




