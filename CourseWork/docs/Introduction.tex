\section*{Введение}
\addcontentsline{toc}{section}{Введение}
\label{sec:Introduction} \index{Introduction}
\large 

Проблема изоморфизма графов в настоящее время приобрела статус одной из самых сложных и важных с теоретической точки зрения задач комбинаторной математики. Известно, что решение задачи нахождения изоморфного подграфа с точки зрения алгоритмической сложности позволило бы ответить на вопрос о совпадении или различии классов P и NP. В последние годы сформировались два направления изучения и решения данной проблемы. Первое направление - теоретическое, в котором проблема изоморфизма рассматривается с позиций соременной теории сложности алгоритмов и вычислений, а второе - сугубо практическое, предполагающее разработку алгоритмов, решающих задачу изоморфизма графов за "практически приемлемое" время.

Первые серьёзные результаты в теоретическом направлении появились в 50-х и 60-х годах прошлого века. Для нескольких семейств графов была установлена полиномиальная сложность нахождения изоморфизма. В 70-х годах, особенно после выхода в свет замечательной монографии Нормана Биггса "Алгебраическая теория графов", к решению этой задачи подключились крупные специалисты в области линейной алгебры, теории групп и ряда других направлений общей алгебры. Результаты не замедлили сказаться - открыто много новых интересных алгебраических свойств графов. В 1979 году Егорову В.Н. удалось получить положительное решение гипотезы Адама об изоморфизме графов с циркулянтными матрицами смежности вершин порядка свободного от квадратов [1]. В 80-х годах усилиями большой группы математиков был получен наилучший на настоящие момент результат о том, что для графов с ограниченной валентностью вершин задача распознавания изоморфизма имеет полиномиальную сложность [2].

Известно, что эта тематика в разных модификациях имеет широкий спектр приложений в таких областях:
\begin{enumerate}
\item Криптография.
\item Теория кодирования.
\item Информационные технологии.
\item Химия.
\item Биология.
\end{enumerate}

Следует отметить, что и здесь сложились два принципиально различных подхода. Первый подход использует понятие инвариантов графа, т.е. таких функций от вершин графа или от определённых структур графа, включая и сам граф, которые не меняют своего значения при перестановках вершин графа. Примеров таких инвариантов можно привести достаточно много, причём с их помощью часто удавалось решать самые разнообразные частные задачи. Однако, полной системы инвариантов, которая позволяла бы определить наличие или отсутствие изоморфизма до настоящего времени получить не удалось. Кроме того, слабостью этого подхода является то, что эти инварианты бесполезно применять в случае, если рассматриваемые графы имеют заведомо транзитивные группы автоморфизмов.

Второй подход не связан с вычислением различных характеристик графов. Предлагаемые алгоритмы являются итерационными и определяют наличие или отсутствие изоморфизма методом направленного перебора, основной идеей которого является построение частичных отображений с последующим промежуточным критерием. Понятно, что это одна из версий метода ветвей и границ. Пожалуй, первое упоминание о таком алгоритме содержится в [3] (стр. 397-402). В настоящее время имеется несколько алгоритмов такого рода, которые авторами рекламируются как "эффективные" (например, [4]).



В дипломной работе рассматривается итерационный алгоритм поиска автоморфизмов и изоморфизмов графов.

Поиск автоморфизмов и изоморфизмов алгебраических систем, в том числе графов, является известной задачей. Опубликовано множество книг и научных статей, посвященных данной проблеме. Из существующих алгоритмов на данный момент стоит выделить "кратко об алг. Лакса (про ограниченную Валентность)"{}, "кратко об алг. в случае циркулянтной матрицы смежности"{}.

Как выяснилось, эффективного (полиномиального) алгоритма, который был бы универсальным, т.е. применим к любым графам, сейчас не существует. На основании этого было решено исследовать итерационный алгоритм, предложенный в статье В.Н.Егорова [1].

Работа разбита на следующие пункты:
\begin{enumerate}
\item Постановка задачи.
\item Основные определения и обозначения: приводятся основные понятия, используемые в ходе работы.
\item Структура алгоритма.
\item Исследование и модернизация алгоритма: в данном пункте описаны полученные оценки сложности алгоритма, а также способы модернизации, которые удалось найти в ходе исследования. Особое внимание уделяется исследованию расспараллеливания вычислений программной реализации.
\item Практическое применение: рассказывается о различных применениях алгоритма для графов и некоторых других комбинаторных объектов. В том числе, рассматриваются способы использования алгоритма для задачи Коши.
\item Разработка программных модулей: рассматриваются ключевые моменты, на которые стоило уделить особое внимание при реализации програмных продуктов. В результате получены 2 программы: упрощенный вариант с графическим интерфейсом для использования на графах с небольшим количеством вершин; вариант реализации со всеми модернизациями, упомянутыми во 2-ом пункте, для более удобного сбора информации и исследования эффективности.
\item Результаты опробывания программ: в этом пункте приводится анализ результатов, полученных при тестировании разработанных программ на различных данных, также подробно проанализированы результаты запусков программы на суперкомьпьютере Ломоносов.
\end{enumerate}