\section{Результаты опробывания программ}
\label{sec:Results_8} \index{Results_8}
\large 


Получена приблизительная сложность работы усовершенствованного алгоритма и в случае случайного графа, модернизация алгоритма дает неплохой результат на небольших размерах. Но эффективность стремительно падает, при увеличении количества вершин графа.

Время работы программы в зависимости от размера графа на среднем по мощности компьютере (затраты оперативной памяти $\thickapprox 2\times 10^9$ байт):
\begin{enumerate}
\item  До 200 вершин - $<$ 1 секунды.
\item  В пределах 300 вершин - около 2 секунд.
\item  400 — 500 вершин - до 20 секунд.
\end{enumerate}




Приблизительная оценка времени работы программы на мощной вычислительной технике:
оперативная память $10^{14}$ байт,
количество операций в секунду $10^{15}$ [5].
\begin{enumerate}
\item 1000 вершин - $<$ 1 секунды  ($10^{12}$  операций).
\item 3000 вершин - $<$ 5 секунд ($5\times 10^{15}$ операций).
\item 10000 вершин - около $2,5$ часов ($10^{19}$ операций).
\item 100000 вершин - около 320000 лет ($10 ^{28}$ операций).
\end{enumerate}
В последнем случае требуется $10^{21}$ байт оперативной памяти.

