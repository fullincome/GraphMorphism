\section{Исследование и модернизация алгоритма}
\label{sec:AlgoModern} \index{AlgoModern}
\large


\subsection{Исследование}
Алгоритм применим для любых графов, но для удобство рассматриваются случайные графы, матрицы смежности которых состоят из нулей и едениц.
В этом случае получена оценка сложности алгоритма в среднем.

Так как граф случайный, элементы матрицы смежности графа равны $1$ или $0$ с вероятностью $\frac{1}{2}$. Необходимо вычислить наиболее вероятные размеры для каждого множества $M'_k$.

Если рассмотреть построение множеств, не учитывая промежуточного критерия, то на $k$-ом шаге множество увеличивается \ в $(n-k)$ раз (так как для каждого элемента $g^s_{(k-1)} = (r_1, \ldots, r_{k-1})$ добавляется $r_k$ из оставшихся $(n-k)$ вариантов): $|M_{k+1}| = |M_{k}|(n-k)= n(n-1)\ldots(n-k)$.

С учетом критерия получается:

\begin{enumerate}
\item Из построения множества $M'_1$ следует $|M'_1|\thickapprox\frac{|M_1|}{2}$ (элемент $a_{11}$ = $a_{r_1r_1}$ с вероятностью $\frac{1}{2}$).
\item На $(k+1)$-ом шаге требуется совпадение $((k+1) + (k+1) - 1)$ элементов. Так как граф случайный (матрица с равновероятным распределением $0$ и $1$), то на этом шаге $|M'_{k+1}| = \frac{1}{2^{2m+1}}|M_k|$ элементов.
\end{enumerate}

 Учитывая построение $\{M'_k\}$ и пункты 1, 2 получим
 
 $|M'_{k+1}| \thickapprox \frac{n!}{(n-k-1)!~2^{(k+1)^2}} \thickapprox \frac{1}{e^{k+1}} \frac{n^{n+1/2}}{(n-k-1)^{(n-k-1/2)}~2^{(k+1)^2}}$. Последнее равенство получено используя формулу Стирлинга: $n!\thickapprox \sqrt{2\pi n}(\frac{n}{e})^n$, при большом $n$.
 


Далее необходимо рассмотреть последовательность $\{|M'_{k+1}|\}$. 

$|M'_{k+1}| = \frac{n(n-1)\ldots(n-k)}{2^{(k+1)^2}} \le \frac{n^{k+1}}{2^{(k+1)^2}}$

Равенство $n^{k+1} = 2^{(k+1)^2}$ означает, что в множестве осталось небольшое количество элементов. Данное равенство далее будет называться стабилизацией последовательности $\{M'_k\}$, а $k$ - значением стабилизации.

После решения данного равенства, получается, что стабилизация наступает при $k \thickapprox \log _2(n) $.
Это означает, что с вероятностью близкой к единице (для случайного графа) уже на $k = \lceil\log _2(n) \rceil$ шаге мы обнаружим отсутствие или наличие автоморфизма (в статье это выдвигается как тезис).

Вычисление номера множества $k$ в последовательности $\{|M'_{k}|\}$, соответствующий самому большому множеству:

Пусть $|M'_{k+1}|:$ $|M'_{k+1}| = f(k+1)$, 

$f'(k+1) = \frac{\ln(n)n^{k+1}2^{(k+1)^2}-(2k+2)\ln(2)2^{(k+1)^2}n^{k+1}}{(2^{(m+1)^2})^2}$. 

Получается уравнение:

$\ln(n) - (2k+2)\ln(2) = 0\ \Rrightarrow \  k+1 \thickapprox \frac{5}{7}\ln(n) - 1$. 

Так как $k$ и $n$ целые, то 

$k + 1 = \lceil(\frac{5}{7}\ln(n) - 1)\rceil)$.

Необходимо отметить, что в тот момент, когда $k \thickapprox \log _2(n)$, можно судить о том, существуют автоморфизмы в графе или нет. Если размер множества $M_k = 1$ (только тождественная подстановка), можно утверждать с вероятностью близкой к 1, что автоморфизмов, кроме тривиальной подстановки, не существует. Если $M_k > 1$, то, наоборот, с вероятностью близкой к 1 автоморфизм существует. Данный факт позволяет сэкономить память и уменьшить количество операций при поиске автоморфизмов в тех графах, в которых они существуют.

Опираясь на вышесказанное, получены оценки:
\begin{itemize}
\item значение стабилизации 

$k \thickapprox \log _2(n)$
\item номер наибольшего по количеству элементов множества 

$k \thickapprox \frac{5}{7}\ln(n)$
\item ограничение на размер множества 

$|M'_k| = \frac{n(n-1)\ldots(n-k-1)}{2^{(k)^2}} \le \frac{n^{k}}{2^{(k)^2}}$
\item приблизительное количество операций в секунду:

$\frac{n^{(5/7) \ln(n)}}{2^{((5/7)\ln(n))^2}}\times n(2(\frac{5}{7}\ln(n)) + 1)\times \log_2(n)$

\item затраты оперативной памяти: 

$2 \times \frac{n^{(5/7) \ln(n)}}{2^{((5/7)\ln(n))^2}} \times \log_2(n)$
\end{itemize}


\subsection{Модернизация}
На основе результатов тестирования программы и анализа выяснилось, что эффективность алгоритма тесно связана с тем, как нумеруются вершины графа. Другими словами, время работы программы зависит от того, какой матрицей смежности (из многих) представляется граф.

Модернизация заключается в том, что на каждом этапе можно требовать, чтобы мощность множества частичных отображений была минимальна. Однако представить матрицу смежности нужным образом не представляется возможным при больших размерах, и время работы, затрачеваемой на это, превышает время работы алгоритма. Например, при $n = 100$ потребуется всего лишь 100 операций, чтобы выяснить, с какой вершины эффективнее всего начать отсчет на первой итерации. Но для того, чтобы получить выгоду на второй итерации, уже требуется более 10 000 операций [3]. Поэтому оптимальным является уменьшить только начальное множество частичных отображений.

Для получения первого множества, мощность которого будет наименьшей, в
изначальной матрице меняется порядок строк/столбцов (переименование вершин), а
именно, необходимо поменять строки и соответствующие им столбцы таким образом,
чтобы на месте первого элемента главной диагонали стояло то значение, которое встречается меньше всех других на диагонали. То есть, если на главной диагонали $70\%$ нулей и $30\%$ единиц,
необходимо поместить на первую позицию единицу. Эффективность данной модификации
исходит из того, что первая итерация алгоритма составляет множество из тех номеров
строк (столбцов), в которых значение на главной диагонали совпадает с первым
элементом главной диагонали. Значит, выбрав на эту позицию наименьший по количеству встречаний на диагонали
элемент, получается наименьшее по мощности множество.

Оценки получены для случайных матриц (вероятность нуля и единицы в каждой позиции
одинакова и равна $\frac{1}{2}$).
Предположим, что в матрице $n\times n$ на главной диагонали находится $k$ нулей, где $k \leq \frac{1}{2} n$ (если количество нулей больше половины, то за $k$ обозначается количество единиц).
Тогда, если на первое место главной диагонали выбирать элемент случайным образом,
получим, что математическое ожидание размера полученного множества будет $\frac{k}{n}*k + \frac{n-k}{n}*(n-k)$.
В модифицированном алгоритме всегда будет получаться $k$. Таким образом,
улучшение составляет $\frac{\frac{k}{n} * k + \frac{(n-k)}{n} * (n-k)} {k} = 2 \frac{k}{n} + \frac{n}{k} - 2$ раз. В случае диагонали,
состоящей из одних нулей (единиц), то есть k = 0, улучшение равняется 1 (мнощьность
множеств одинакова).
Для получения полной оценки необходимо посчитать матожидание улучшения для
произвольного числа нулей и единиц на главной диагонали. Вероятноть $k$ нулей
составляет $\frac{C_n^k}{2^n}$. Тогда матожидание улучшения $\frac{1}{2^{n-1}} + 2 \sum\limits_{i=1}^{\lceil\frac{n}{2}\rceil} \frac{C_n^k}{2^n} * (2\frac{k}{n} + \frac{n}{k} - 2)$.
Далее в таблице приведены значения для различных $n$:

\begin{tabular}[t]{||l|l||}
\hline
количество вершин графа & коэф. уменьшения мощности \\
\hline
4 &  2.12500 \\
8 & 2.11198 \\
14 & 1.69565 \\
20 & 1.51723 \\
50 & 1.27697 \\
100 & 1.18312 \\
200 & 1.12400 \\
\hline
\end{tabular}





